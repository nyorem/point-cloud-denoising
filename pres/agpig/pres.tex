% {{{1 PRELUDE
\documentclass{beamer}

\usepackage[francais]{babel}
% \usepackage[english]{babel}
\usepackage[utf8]{inputenc}
\usepackage[T1]{fontenc}

\usetheme{Frankfurt}

\AtBeginSection[]
{
    \begin{frame}
        \frametitle{Sommaire}
        \tableofcontents[currentsection, hideothersubsections]
    \end{frame}
}

\title[Lissage adaptatif de nuage de points]{Lissage adaptatif de nuage de
    points}
\author[MEYRON Jocelyn]{MEYRON Jocelyn\\\scriptsize{Encadré par:\\ATTALI Dominique\\MÉRIGOT
    Quentin}}
\institute{GIPSA-lab}
\date{\today}

\begin{document}

\begin{frame}
    \titlepage
\end{frame}

\begin{frame}
    \tableofcontents
\end{frame}

% {{{1 INTRODUCTION
\section{Introduction}

\begin{frame}
    \frametitle{Introduction}
    % \framesubtitle{Flot de courbure moyenne}

    Flot de courbure moyenne:
    \begin{itemize}
        \item Faire évoluer une surface en bougeant chaque point: dans la
            direction de la normale et d'une quantité égale à la courbure
            moyenne en ce point
        \item Propriétés lissantes
    \end{itemize}

    \begin{figure}
        \centering
        \includegraphics[scale=0.27]{img/mean-curvature-flow-rabbit}
        \caption{Flot de courbure moyenne sur une surface bruitée}
    \end{figure}
\end{frame}

\begin{frame}
    \frametitle{Introduction}
    % \framesubtitle{Objectifs}

    \emph{Objectifs}
    \begin{enumerate}
        \item Flot de courbure moyenne sur des nuages de points: calculer une
            fonctionnelle de l'union des boules (aire, périmètre du bord) et
            faire une descente de gradient
        \item Flot de courbure moyenne anisotrope: remplacer l'union des boules
            par la somme de Minkowski avec un polyèdre convexe
    \end{enumerate}

    \emph{Applications:}
    \begin{enumerate}
        \item Débruitage
        \item Simulation de croissance de cristaux
    \end{enumerate}
\end{frame}

\begin{frame}
    \frametitle{Introduction}
    % \framesubtitle{Résultats}

    Approximation de la courbure moyenne
    \begin{theorem}[Lien gradient et courbure moyenne]
        Soit une surface $ M $ de $ \mathbb{R}^3 $ et soit $ P $ un
        $\epsilon$-échantillon de $ M $, on a:
        $$ \frac{\partial A}{\partial p} \approx 4 r \vec{\kappa}(p) Vol(V(p, P) \cap
        M) $$
        où $ A $  = volume de l'union des boules $ B(p, r) $ et $ V(p, P) $ =
        cellule de Voronoi de $ p $.
    \end{theorem}
\end{frame}

% {{{1 2D
\section{Cas 2D}

\subsection{Problème}
\begin{frame}
    \frametitle{Problème}
    % \framesubtitle{Flot de courbure moyenne}

    Flot de courbure moyenne sur des nuages de points: descente de gradient
    \begin{itemize}
        \item Calcul du volume (ou du périmètre du bord) de l'union de boules
        \item Calcul du gradient
        \item Déplacement des points (Euler explicite): $ p'_i = p_i - \tau \nabla f (p_i) $ où $
            \tau $ est une constante
    \end{itemize}

    \begin{figure}
        \centering
        \includegraphics[scale=0.28]{img/ellipse-balls-15}
        \caption{Union de boules autour d'une ellipse}
    \end{figure}
\end{frame}

\begin{frame}
    \frametitle{Problème}
    % \framesubtitle{Outils}

    Outils:
    \begin{itemize}
        \item CGAL: bibliothèque C++ de géométrie algorithmique
        \item Différentiation automatique: calcul de dérivées de fonctions (au
            sens informatique du terme) en surchargeant le type de nombre,
            remplacement de $ x $ par $ x + y\epsilon $ avec $ \epsilon^2 = 0 $
        \item Qt
    \end{itemize}
\end{frame}

\subsection{Estimation de courbure moyenne}
\begin{frame}
    \frametitle{Estimation de courbure moyenne}
    % \framesubtitle{Comment?}

    Comment?
    \begin{itemize}
        \item Norme du gradient $ \propto $ courbure moyenne
        \item Moyen d'estimer la courbure moyenne d'un nuage de points
            échantillonnant une surface
    \end{itemize}
\end{frame}

\begin{frame}
    \frametitle{Estimation de courbure moyenne}
    % \framesubtitle{Exemples}

    Exemples: courbure d'une ellipse
    \begin{figure}
        \centering
        \includegraphics[scale=0.3]{img/curvature-ellipse-200-15-area}
        \includegraphics[scale=0.3]{img/curvature-ellipse-200-15-perimeter}
        \caption{Aire / périmètre du bord}
    \end{figure}
\end{frame}

\subsection{Flot de courbure moyenne}
\begin{frame}
    \frametitle{Flot de courbure moyenne}
    % \framesubtitle{Exemples}

    Exemples: flot d'une ellipse
    \begin{figure}
        \centering
        \includegraphics[scale=0.3]{img/ellipse-100-01-15}
        \includegraphics[scale=0.3]{img/ellipse-100-01-15-100}
        \caption{0 / 100 itérations}
    \end{figure}
\end{frame}

% {{{1 3D
\section{Cas 3D}

\subsection{Problème}
\begin{frame}
    \frametitle{Problème}
    % \framesubtitle{Flot anisotrope}

    Flot anisotrope:
    \begin{itemize}
        \item Somme de Minkowski avec un polyèdre convexe
        \item Directions privilégiées
    \end{itemize}

    Différentes méthodes:
    \begin{itemize}
        \item Naïve (approchée)
        \item Formules d'inclusion-exclusion (approchée)
        \item Arrangements 3D (exacte?)
    \end{itemize}
\end{frame}

\subsection{Exemples}
\begin{frame}
    \frametitle{Exemples}
    % \framesubtitle{Polyèdres}

    Polyèdres:
    \begin{itemize}
        \item Discrétisation d'une sphère: estimation de normales
        \item Cube (Norme $ L_{\infty} $)
        \item Bipyramide (Norme $ L_1 $)
    \end{itemize}
\end{frame}

\subsection{Flot}
\begin{frame}
    \frametitle{Estimation de normales}
    % \framesubtitle{Exemples}

    Exemples:
    \begin{figure}
        \centering
        \includegraphics[scale=0.25]{img/sphere-1000}
        \includegraphics[scale=0.25]{img/sphere-sphere-1000-05}
        \caption{Estimation de normales sur une sphère}
    \end{figure}
\end{frame}

\begin{frame}
    \frametitle{Flot anisotrope}
    % \framesubtitle{Exemples}

    Exemples:
    % TODO
\end{frame}

% {{{1 PERSPECTIVES
\section{Perspectives}

\begin{frame}
    \frametitle{Perspectives}
    % \framesubtitle{Sous-titre}

    \begin{itemize}
        \item Améliorer calcul en 3D (utilisation d'arrangements 3D)
    \end{itemize}
\end{frame}

\plain{Merci!}

\end{document}

% vim: set spelllang=fr :
