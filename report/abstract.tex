\renewcommand{\abstracttextfont}{\normalfont}
\abstractintoc

% English
\begin{abstract}

% TODO
During this internship, we were interested in the denoising / smoothing of point
clouds sampled on an unknown surface. For doing that, we will use a mean
curvature flow approach: we will move the points while minimizing an energy, the
area of the underlying surface. So, we will need a way to approximate the area
of a surface by only knowing points sampled on it, we will use the volume of a
union of balls centered on the point cloud.  We will also be interested in
another kind of flow: an anisotropic one. The magnitude of the smoothing will be
more important in some directions dictated by the choice of the polyhedron. The
idea is to replace the union of balls by a union of convex polyhedra and do the
same work.

This report is divided as follows: firstly, we will give a detailed introduction
about point cloud smoothing: existing techniques and why using a mean curvature
flow can be interesting. Then, we will study the two dimensional case. We will
show that this technique can be used to simulate a discrete mean curvature flow
and so smooth point clouds. The technique can also be used to estimate the mean
curvature. Many examples will be studied. Furthermore, we will continue by
looking at the 3D case where we will study the volume of a union of convex
polyhedra. Several theoretical and practical issues will arise. We will explain
the choices that were mode to solve these issues. Finally, we will look at the
theory behind our approach.

\end{abstract}

\newpage
\abstractintoc
\renewcommand\abstractname{R\'{e}sum\'{e}}

% Français
\begin{abstract} \selectlanguage{french}

% TODO
Durant ce stage, nous nous sommes intéressés au débruitage / lissage de nuages
de points échantillonnés sur une surface inconnue. Pour faire cela, nous avons
utilisé une approche basée sur le flot de courbure moyenne: nous faisons évoluer
le nuage de points tout en minimisant une énergie, l'aire de la surface
sous-jacente. Donc, nous avons besoin d'un moyen pour approcher l'aire de la
surface en connaissant seulement des points échantillonnés sur celle-ci, nous
avons utilisé le volume de l'union des boules centrées sur le nuage de point.
Nous nous somme aussi intéressé à un autre genre de flot: un flot anisotropique.
La magnitude du flot sera plus importante dans certaines directions privilégiées
dictées par le choix du polyèdre. L'idée est de remplacer l'union des boules par
une union de polyèdres convexes et de faire le même type de travail.

Ce rapport est structuré de la manière suivante: tout d'abord, nous donnerons
une introduction détaillée sur les techniques existantes de lissage de nuages de
points et pourquoi il est intéressant d'avoir choisi une approche basée sur le
flot de courbure moyenne. Ensuite, nous étudierons le cas bidimensionnelle. Nous
montrerons que cette technique peut être utilisée pour simuler un flot de
courbure moyenne discret et donc de lisser des nuages de points. Cette technique
peut aussi être utilisée pour estimer la courbure moyenne. Beaucoup d'exemples
illustreront ces résultats. Par la suite, nous nous intéresserons au cas 3D pour
lequel nous étudierons le volume d'une union de polyèdres convexes. Nous serons
confrontés à des difficultés aussi bien théoriques que pratiques. Nous
expliquerons alors les choix qui ont été faits pour résoudre les problèmes
rencontrés. Enfin, nous nous intéresserons à la théorie utilisée derrière notre
approche.

\end{abstract}

\selectlanguage{english}

% vim: set spelllang=en :
