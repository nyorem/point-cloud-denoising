\renewcommand{\abstracttextfont}{\normalfont}
\abstractintoc

% English
\begin{abstract}

% TODO
During this internship, we were interested in the denoising / smoothing of point
clouds and more specifically in an adaptive one: the magnitude of the smoothing
will be more important in some privileged directions. In order to do that, we
will use a mean curvature flow based method: the idea is to move the point cloud
while minimizing an energy. This energy will be related to the Minkowski sum of
the point cloud with a convex polyhedron. The choice of this polyhedron will
determine the privileged directions of the smoothing.

This report is divided as follows: firstly, we will give a detailed introduction
on the mean curvature flow and we will prove some properties (valid in any dimension
$d$) which will be used throughout the rest of the report like the convergence
of the gradient of the energy towards the mean curvature vector.

Secondly, we will study the two dimensional case with the $r$-offset
of a point cloud. We will prove that this technique can be used to simulate a
discrete mean curvature flow. Many examples will be studied.

We will continue by looking at the 3D case where we will study the Minkowski sum
of a 3D point cloud with a convex polyhedron. Several theoretical and practical
issues will arise. We will deal with these issues and explain the choices that
were made.

\end{abstract}

\newpage
\abstractintoc
\renewcommand\abstractname{R\'{e}sum\'{e}}

% Français
\begin{abstract} \selectlanguage{french}

% TODO
Durant ce stage, nous nous sommes intéressés au débruitage / lissage de nuage de
points et plus précisément à un lissage adaptatif: le lissage sera plus
important dans certaines directions privilégiées. Les techniques utilisées
seront basées sur le flot de courbure moyenne: l'idée est de faire évoluer le
nuage de points tout en minimisant une énergie. Cette dernière sera liée à la
somme de Minkowski du nuage de points avec un polyèdre convexe. Le choix de ce
polyèdre déterminera les directions privilégiées du lissage.

Ce rapport est structuré de la manière suivante: tout d'abord, nous donnerons
une introduction détaillée sur le flot de courbure moyenne et nous démontrerons
des propriétés (valable en dimension quelconque) que nous utiliserons tout au
long du rapport comme la convergence du gradient de l'énergie considérée vers le
vecteur courbure moyenne.

Ensuite, nous étudierons le cas bidimensionnelle en choisissant le $r$-offset
d'un nuage de points. Nous montrerons que cette technique peut être utilisée
pour simuler un flot de courbure moyenne discret. Beaucoup d'exemples
illustreront ces résultats.

Par la suite, nous nous intéresserons au cas 3D pour lequel nous étudierons la
somme de Minkowski d'un nuage de points avec un polyèdre convexe. Nous serons
confrontés à des difficultés aussi bien théoriques que pratiques. Nous
expliquerons alors les choix qui ont été faits pour résoudre les problèmes
rencontrés.

\end{abstract}

\selectlanguage{english}

% vim: set spelllang=en :
