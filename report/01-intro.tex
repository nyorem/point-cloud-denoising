\chapter{Introduction}

% TODO

Point cloud smoothing is a well-studied problem: numerous methods exists and
have been proven to work more or less nicely on different kinds of data
(presence of noise...). In this report, we want to tackle the problem of
smoothing while preserving informations such as small details.

More precisely, given a set of points, we want to be able to smooth it by taking
into account the different scales present in it: we want to be able to control
the smoothing such that it is more important in some directions. An example
where this kind of smoothing could be interesting is when we build a 3D scan of
an object: we want to be able to smooth the point cloud while preserving details
(the signature of the author for example).

A lot of smoothing algorithms already exist, some are related to Computer
Vision, some are more based on Computational Geometry techniques:
\begin{itemize}
    \item Gaussian / Laplacian smoothing and all the related filters: adaptive
        filter...
    \item Jet smoothing: a jet is a truncated Taylor expansion. Such jets are
        fitted around points. Jet smoothing operates by projecting the input
        points on an estimated smooth parametric surface (the so-called jet
        surface). Jets are good because they intrinsically contain
        differential information such as normal, curvature...
\end{itemize}

In this report, we will focus on another algorithm which was proposed by
\cite{chambolle2012nonlocal}. Basically, it is a mean curvature flow under an
energy minimization. This energy is related to a non-local curvature.

More formally, let's suppose we have a point cloud in $ \mathbb{R}^d $ made of $
N $ points : $ p_1, \ldots, p_N $, we want to apply an adaptive smoothing on it.
Firstly, we will compute the energy that we want to minimize: $ E(P) = Vol(P + r
K) $ where $ K $ is a chosen convex polyhedron, $ r \in \mathbb{R} $ and $ (+) $
is the Minkowski sum. If we choose $ K = B(0, 1) $ an euclidean ball, then $ P +
rK $ is the $ r $-offset of P and $ E(P) $ is a measure of a union of balls
(like its volume, or the area of its boundary). In this case, the $ r $-offset
will contain intrinsic informations about the point set like normal direction or
curvature approximation.

Secondly, we will need to compute the gradient of this energy. For doing that,
there are multiple choices: use analytical formulae (which exist at least in
2D), use approximations (finite differences) or use a technique called automatic
differentiation that allows use to differentiate any function by changing the
number type and so increasing the complexity a little bit. We will try to use,
in most cases, the automatic differentiation method since it allows us to have
better results than approximate methods without being too painful to implement.

Finally, we will do one step of the flow: move our point set in the direction
which will be computed as the result of a gradient descent algorithm applied on
the point set. The weighting of the gradients will be an important question.

The advantage of such algorithm resides in its anisotropy: since the polyhedron
$ K $ may be any convex polyhedron, it can privilege some directions in the
space and so preserve details in some directions.

We will test this algorithm using multiple polyhedrons, values of $ r
$, point sets with a small / high amount of noise to see if the algorithm is
noise-resilient or not.

% vim: set spelllang=en :
